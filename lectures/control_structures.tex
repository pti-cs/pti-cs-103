\chapter{Control Structures}

So far, we have seen programs in which each statement is executed in order, one by one. Today we will learn about \emph{conditionals}, which allow us to execute statements depending on certain conditions. This is our first exposure to the idea of \emph{control flow}, which refers to the order (or sequence) in which statements of a program are executed.

In this chapter, we will first learn about \ic{if} statements, which allow us to write programs based on conditions. Then we will learn about \ic{else} statements. We will combine these \ic{if} and \ic{else} statements into more complex nested structures, and then finally learn about \ic{else if} statements. Lastly, this chapter ends with a list a common mistakes to avoid when writing conditionals.

\section{The \ic{if} statement}

\begin{definition}
An \emph{if statement} is a \emph{conditional statement} that consists of the reserved keyword \ic{if}, followed by a boolean expression enclosed in parentheses, followed by a statement typically enclosed in curly braces. If the boolean expression (or \emph{condition}) evaluates to true, the statement is executed. Otherwise, it is skipped.
\end{definition}

An \ic{if} statement allows us to write programs that decide whether to execute a particular statement, based on a boolean expression which we call the \emph{condition}. Below is an example of a simple \ic{if} statement:

\begin{code}
if (count > 20) {
    System.out.println("Count exceeded");
}
\end{code}

\noindent The condition in this example is \ic{count > 20}. It is a boolean expression that evaluates to either true or false. That is, \ic{count} is either greater than \ic{20} or not. If it is, ``Count exceeded" is printed. Otherwise, the \ic{println} statement is skipped. 

\begin{example}
What does the following piece of code print if \ic{x} is 101? What about if \ic{x} is 200? What about if \ic{x} is 8?

\begin{code}
if (x > 100) {
    System.out.println("Big number!");
} 
System.out.println("Hi there");
if (x % 2 == 0) { // checks if a number is even*
    System.out.println("Even number!");
}
\end{code}

* Recall that the modulo operator (\%) computes the remainder of some number. By checking if the remainder of some number when divided by 2 is 0, we are checking if that number is even. \\

\emph{Answer}: If \ic{x} is 101, the first boolean expression will evaluate to true, so the program will print ``Big number!" Then the program will print ``Hi there" regardless of any condition. Finally, since 101 is not even, the program will not print ``Even number!"

Using the same reasoning to trace the execution with \ic{x} as 200, the following statements will be printed: ``Big number! Hi there! Even number!"

When \ic{x} is 8, the output is ``Hi there! Even number!"
\end{example}

\begin{example}
How would you fill in the boolean expression below to take the absolute value of an integer \ic{x}? (Hint: to take an absolute value of a negative number, you must negate it).

\begin{code}
if (/* Insert boolean expression here */) {
    x = -x;
} 
\end{code}

\emph{Answer}: Insert the boolean expression x $<$ 0.
\end{example}

\section{The \ic{else} statement}
Sometimes we want to do one thing if a condition is true and another thing if that condition is false. We can add an \ic{else} clause to an \ic{if} statement to handle this kind of situation. Below is an example of a simple \ic{if-else} statement:

\begin{code}
if (price > 20) {
    System.out.println("Too expensive");
} else {
    System.out.println("Affordable");
}
\end{code}

\noindent This example prints either ``Too expensive" or ``Affordable", depending on whether \ic{price} is greater than 20. Only one or the other is ever executed, never both. This is because boolean conditions only evaluate to either true or false.

Note that it is not possible to have an \ic{else} statement without having a corresponding \ic{if} statement. Any \ic{else} statement must be attached to a preceding \ic{if} statement. 

\begin{example}
What does the following piece of code do?

\begin{code}
if (balance <= 0) {
    System.out.println("Unable to withdraw");
} else {
    System.out.println("Withdraw successful");
}
\end{code}

\emph{Answer}: It prints ``Unable to withdraw" if balance is 0 or lower. Otherwise, it prints ``Withdraw successful".
\end{example}

\begin{example}
Rewrite the code below using an \ic{else} statement.

\begin{code}
if (rating >= 4) {
    System.out.println("Would recommend");
} if (rating < 4) {
    System.out.println("Would not recommend")
}
\end{code}

\emph{Answer}: Since the second boolean expression is the negation of the first boolean expression, we can replace the second \ic{if} statement with an \ic{else}.
\begin{code}
if (rating >= 4) {
    System.out.println("Would recommend");
} else {
    System.out.println("Would not recommend")
}
\end{code}
\end{example}

\begin{example}
Does the following code compile?

\begin{code}
else {
    System.out.println("Hello");
}
\end{code}

\emph{Answer}: No, since an \ic{else} statement must have a corresponding \ic{if} statement.
\end{example}

\begin{example}
Does the following code compile?

\begin{code}
if (duration > 60) {
    System.out.println("Sorry, that's too long");
} else {
    System.out.println("I can fit that in my schedule");
} else {
    System.out.println("Let me know when you are free")
}
\end{code}

\emph{Answer}: No, since an \ic{if} statement may only have at most 1 corresponding \ic{else} statement.
\end{example}

\section{Nested conditionals}
It is possible to combine \ic{if} and \ic{else} statements in more interesting ways, by nesting them. Consider the following example:

\begin{code}
if (x < y) {
    System.out.println("x is less than y");
} else {
    if (x > y) {
        System.out.println("x is greater than y");
    } else {
        System.out.println("x is equal to y");
    }
}
\end{code}

\noindent This example correctly prints out the relationship between two integer variables \ic{x} and \ic{y} using nested conditionals. No matter what the values of \ic{x} and \ic{y} are, only a single statement will be printed. 

\begin{example}
Fill in each blank below with \ic{if} or \ic{else} to describe a person's height.

\begin{code}
/*__________*/ (height < 60) {
    System.out.println("Relatively short")
} /*__________*/ {
    /*__________*/ (height > 72) {
        System.out.println("Relatively tall");
    } /*__________*/ {
        System.out.println("Pretty average");
    }
}
\end{code}

\emph{Answer}: The four blanks should contain \ic{if}, \ic{else}, \ic{if}, and \ic{else}, in that order.
\end{example}

\begin{example}
Given integer variables \ic{a} and \ic{b} and the following piece of code, what value gets stored in variable \ic{c}?

\begin{code}
int c;
if (a > b) {
    c = a;
} else {
    if (b > a) {
        c = b;
    } else {
        c = 0;
    }
}
\end{code}

\emph{Answer}: The larger of the two variables, \ic{a} and \ic{b}, gets stored in \ic{c}. If \ic{a} and \ic{b} are equal, 0 gets stored in \ic{c}.
\end{example}

\begin{example}
Imagine that the following program represents the person's reaction to different types of food. Given a pizza (\ic{green} = false, \ic{bitter} = false, \ic{warm} = true, \ic{cheesy} = true), how will this person respond? What about when given a kiwi (\ic{green} = true, \ic{bitter} = false, \ic{warm} = false, \ic{cheesy} = false)? What about a strawberry (\ic{green} = false, \ic{bitter} = false, \ic{warm} = false, \ic{cheesy} = false)?

\begin{code}
if (green || bitter) {
    System.out.println("No thank you");
} else {
    if (warm && cheesy) {
        System.out.println("Yum, yes please!");
    } else {
        System.out.println("Thank you, I'll take some");
    }
}
\end{code}
\emph{Answer}: The person reponds ``Yum, yes please!" to pizza, ``No thank you" to kiwi, and ``Thank you, I'll take some" to strawberry.
\end{example}

\section{The \ic{else if} statement}
Nested conditionals are useful in many applications, but they can start feeling clunky as the nesting becomes deep. For example, consider the following piece of code:

\begin{code}
if (grade > 90) {
    System.out.println("You earned an A");
} else {
    if (grade > 80) {
        System.out.println("You earned a B");
    } else {
        if (grade > 70) {
            System.out.println("You earned a C");
        } else {
            if (grade > 60) {
                System.out.println("You earned a D");
            } else {
                System.out.println("You earned an F");
            }
        }
    }
}
\end{code}

\noindent To avoid having such deeply nested conditionals, we introduce the \ic{else if} clause. Using \ic{else if} statements, we can rewrite the above program as:

\begin{code}
if (grade > 90) {
    System.out.println("You earned an A");
} else if (grade > 80) {
    System.out.println("You earned a B");
} else if (grade > 70) {
    System.out.println("You earned a C");
} else if (grade > 60) {
    System.out.println("You earned a D");
} else {
    System.out.println("You earned an F");
}
\end{code}

You can imagine an \ic{else if} statement as shorthand for an \ic{if} nested inside an \ic{else}. For example, the following snippets of code behave identically (for any boolean expressions A and B):

\begin{code}
// Version 1 (using nested conditionals)
if (A) {
    System.out.println("A");
} else {
    if (B) {
        System.out.println("B");
    } else {
        System.out.println("C");
    }
}

// Version 2 (equivalent, using else if)
if (A) {
    System.out.println("A");
} else if (B) {
    System.out.println("B");
} else {
    System.out.println("C");
}
\end{code}

It is very common to see a conditional block of code that consists of 1 \ic{if} statement, 0 or more \ic{else if} statements, and 1 \ic{else} statements. In these cases, recall that only one of these statements will ever execute.

\begin{example}
Imagine \ic{A} and \ic{B} are boolean expressions. If both of them evaluate to true, what does the following code print out?

\begin{code}
if (A) {
    System.out.println("apple");
} else if (B) {
    System.out.println("banana");
} else {
    System.out.println("carrot");
} 
System.out.println("dragonfruit");
\end{code}

\emph{Answer}: The code will print ``apple" and then ``dragonfruit". Notice that ``banana" does not get printed even though \ic{B} is true since the entire \ic{if}, \ic{else if}, \ic{else} chain will only ever print one of ``apple", ``banana", and ``carrot". 
\end{example}

\begin{example}
Imagine \ic{X}, \ic{Y}, and \ic{Z} are boolean expressions. If \ic{X} evaluates to true, but \ic{Y} and \ic{Z} evaluate to false, what does the following code print out?

\begin{code}
if (X && Y) {
    System.out.println("xylophone");
} else if (!Z) {
    System.out.println("zoo");
}
\end{code}

\emph{Answer}: This code will print ``zoo". Notice that the boolean expressions used with \ic{if} statements can be complex and contain boolean operators such as \ic{\&\&} and \ic{||} and \ic{!}, as long as the expression evaluates to a true or false.
\end{example}

\section{Curly braces}
So far, we have been using curly braces to enclose each of our \ic{if}, \ic{else if}, and \ic{else} statements. These curly braces can be omitted if there is only a single statement. For example, in the snippet of code below, the first set of curly braces can be omitted, while the second cannot.

\begin{code}
if (guess == answer) {
    System.out.println("You guessed correctly!");
} else {
    System.out.println("Sorry, you guessed incorrectly.");
    System.out.println("The answer was " + answer);
} 
\end{code}

\noindent If all the curly braces were left out (as in the code below), the program would first either print ``You guessed correctly!" or ``Sorry, you guessed incorrectly.", but then also print ``The answer was ..." in all cases, regardless of the condition.

\begin{code}
if (guess == answer)
    System.out.println("You guessed correctly!");
else
    System.out.println("Sorry, you guessed incorrectly.");
    System.out.println("The answer was " + answer); 
    // ^^^ Careful! This is not part of the else clause!
\end{code}

Here it is important to note that whitespace and indentation are ignored by Java. Indentation has no effect on the behavior of a program. Proper indentation is extremely important for human readability. When used incorrectly, however, misleading indentation can result in unexpected behavior.

\begin{example}
Does the following code compile and print the larger of two integers \ic{a} and \ic{b} correctly?

\begin{code}
if (a > b)      System.out.println("a");
else {
System.out.println("b");
}
\end{code}

\emph{Answer}: Yes! Even though the whitespacing and indentation look funny, this piece of code compiles correctly.
\end{example}

\section{Common mistakes}
When writing conditional structures, beware of the following common  mistakes:

\subsection{Forgetting parentheses}
In Java, the parentheses surrounding the boolean expression in conditional structures are required. For example, the following code will not compile.
\begin{code}
if count > 10
    System.out.println("So many!");
\end{code}

\subsection{Accidental semi-colons}
Accidentally semi-colons immediately after the parentheses around the boolean expression in an \ic{if} statement is one of the trickiest mistakes to detect. The following code compiles, but it behaves unexpectedly. 
\begin{code}
if (count > 10);
    System.out.println("So many!");
\end{code}

\noindent The code above is misleading because it is identical to the following:
\begin{code}
if (count > 10) {
    ;
}
System.out.println("So many!");
\end{code}

\noindent The accidental semi-colon is treated as a single, ``do-nothing" statement.

\subsection{Missing curly braces}
As mentioned before, indentation helps make code more readable to humans, but it can also make code more confusing if used incorrectly. It is a common mistake to accidentally forget curly braces.

\begin{code}
if (leaves = 4) {
    System.out.println("You found a four-leaf clover, how lucky!");
} else
    System.out.println("Sorry, not a four-leaf clover.");
    System.out.println("Keep looking!");
\end{code}

\noindent This code above is misleading because it prints ``Keep looking!" regardless of whether a four-leaf clover was found. The writer of this program probably meant to add curly braces around the \ic{else} clause.

\subsection{\ic{else} without an \ic{if}}

Although the following example looks fine at first glance, it does not compile. This is because the \ic{else} clause is nested \emph{inside} the \ic{if} statement instead of acting as an \emph{alternative} option to the \ic{if} statement.

\begin{code}
if (chanceOfRain > 50) {
    System.out.println("Bring an umbrella!");
    else {
        System.out.println("Hm, I don't think it will rain today.")
    }
}
\end{code}

\noindent To fix the code above, move the \ic{else} statement \emph{outside} of the \ic{if} as below:

\begin{code}
if (chanceOfRain > 50) {
    System.out.println("Bring an umbrella!");
} else {
    System.out.println("Hm, I don't think it will rain today.")
}
\end{code}

\subsection{Assignment v.s. equality operator}
It is very easy to mix up the assignment operator (the single equals sign \ic{=}) with the equality operator (the double equals sign \ic{==}). The following code is incorrect and results in a compilation error:

\begin{code}
if (temperature = 0) {
    System.out.println("It's freezing!");
}
\end{code}

\noindent The assignment operator cannot be used here because conditional structures require a boolean expression (i.e. something that evaluates to either true or false). 

\exercisesection

\begin{exercise}
What output is produced by the following code fragment?

\begin{code}
int num = 87;
int max = 25;
if (num >= max*2)
    System.out.println("apple");
    System.out.println("orange");
System.out.println("pear");
\end{code}

TODO: modify this exercise, or make sure we can reference the Java Foundations textbook
\end{exercise}

\begin{exercise}
What output is produced by the following code fragment?

\begin{code}
int limit = 100;
int num1 = 15;
int num2 = 40;
if (limit <= limit) {
    if (num1 == num2)
        System.out.println ("lemon");
    System.out.println ("lime");
}
System.out.println ("grape");
\end{code}

TODO: modify this exercise, or make sure we can reference the Java Foundations textbook
\end{exercise}


\begin{exercise}
Given a day of the week encoded as 0=Sun, 1=Mon, 2=Tue, ...6=Sat, and a boolean indicating if we are on vacation, we need to decide when to set our alarm clock to ring. We want to wake up at ``7:00" on weekdays and ``10:00" on weekends. However, if we are on vacation, it should be ``10:00" on weekdays" and ``off" on weekends. Imagine you are given an integer variable \ic{day} and a boolean variable \ic{vacation}. Write a program that prints out the appropriate alarm time.

TODO: modify this exercise, or make sure we can reference codingbat.com
\end{exercise}

\begin{exercise}
Your cell phone rings. Normally you answer, except in the morning you only answer if it is your mom calling. In all cases, if you are asleep, you do not answer. Given three booleans, \ic{isMorning}, \ic{isMom}, and \ic{isAsleep}, write a program that correctly prints out whether or not you will pick up your phone (``Answer" or ``Do not answer"). 

For example, if \ic{isMorning}, \ic{isMom}, and \ic{isAsleep} are all false, print ``Answer". 

TODO: modify this exercise, or make sure we can reference codingbat.com
\end{exercise}

\begin{exercise}
You have a lottery ticket with a 3-digit number. If your ticket has the number 777, you win \$100. If your ticket has a number with the same 3 digits (like 222, for example), then you win \$10. Otherwise, you don't earn any money. Given three integers, \ic{a}, \ic{b}, and \ic{c}, that represent the digits of your lottery ticket, write a program that prints out how much money you earn.

For example, if your ticket number is 123 (\ic{a} = 1, \ic{b} = 2, \ic{c} = 3), print ``\$0". 

TODO: modify this exercise, or make sure we can reference codingbat.com
\end{exercise}

\referencessection

Computer Science: An Interdisciplinary Approach, Robert Sedgewick and Kevin Wayne.

Lewis, John, Peter DePasquale, and Joseph Chase. Java Foundations: Introduction to Program Design and Data Structures. Addison-Wesley Publishing Company, 2010.

Programming exercises from codingbat.com