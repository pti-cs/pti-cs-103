\chapter{Types and Variables}

We have seen a program that can print a simple ``Hello World'' message to the screen. Next we will learn about basic types of data that can be used in a Java program to perform calculations.

In this chapter, we will first learn about different types of data, focusing on five Java data types: \ic{int}, \ic{double}, \ic{boolean}, \ic{char}, and \ic{String}. Then we will learn about arithmetic, comparison, and logical operators over those types. Finally, we will learn about variables.

\section{Data types}
We work with different types of data all of the time: numbers, text, images, true/false or yes/no information, etc. Since we interact with numbers all the time, we can categorize numeric data in two ways: as (1) integers, which are whole numbers, or as (2) floating point numbers, which are numbers with digits after a decimal point or a fractional component.

Imagine the data you would find on a report card. It would probably have a lot of textual data, such as the student's name and their course names. It would probably also have a lot of numeric data, such as their age or their GPA. \ref{table:report_card} lists examples of information one might find on a report card.

\begin{table}[h!]
\centering
\begin{tabular}{ |c|c|c| }
 \hline
 Item & Example & Type of data \\
 \hline
 \hline
 Student name & John Doe & Text \\
 Course name & Honors Physics & Text \\
 Age & 17 & Integer (whole number) \\
 GPA & 3.75 & Floating point number\\
 \hline
\end{tabular}
\caption{Examples of pieces and types of data typically found on a report card.}
\label{table:report_card}
\end{table}

Now imagine an ice cream menu. It might have pictures of different flavors, along with the names of each flavor. It would probably list prices for different sizes and cones. It might also contain yes/no information, such as whether or not a certain flavor is dairy-free. \autoref{table:ice_cream_menu} lists examples of information one might find on such a menu.

\begin{table}[h!]
\centering
\begin{tabular}{ |c|c|c| }
 \hline
 Item & Example & Type of data \\
 \hline
 \hline
 Ice cream flavor & Mint Chocolate Chip & Text \\
 Price & \$3.99 & Floating point number \\
 Dietary information & Dairy-free & Yes/no \\
 \hline
\end{tabular}
\caption{Examples of pieces and types of data typically found on a menu at an ice cream shop.}
\label{table:ice_cream_menu}
\end{table}

\begin{example}
Take a look at the New Jersey driver's license in \autoref{fig:drivers_license}. List examples of different types of data you can find on it. \\

\noindent \emph{Answer}: Text (name, address), integers (date of birth, height, license number), yes/no (organ donor status).
\end{example}

\begin{figure}[h!]
  \centering
  \includegraphics[width=0.5\textwidth]{images/nj_drivers_license}
  \caption{An example of a New Jersey driver's license. TODO: reference \href{https://www.state.nj.us/mvc/images/vetdefig.jpg}{the NJ state website (link here).}}
  \label{fig:drivers_license}
\end{figure}

We are surrounded by different types of data every day. We intuitively understand that different types of data interact in different ways. For example, you can add up the prices of multiple flavors of ice cream, but you cannot divide a student's name by their GPA. The same is true of data types in a Java program.

\begin{definition}
A \emph{data type} is a set of values and a set of operations defined on them.
\end{definition}

In the following sections, we will learn how Java represents different types of data: numeric data, true/false or yes/no data, and textual data. Specifically, we will focus on five types: \ic{int}, \ic{double}, \ic{boolean}, \ic{char}, and \ic{String}.

\subsection{Numeric data}
Representing numbers on a computer requires physical memory. For this reason, Java has six possible types for numeric data: four integer types (\ic{byte}, \ic{short}, \ic{int}, \ic{long}) and two floating point types (\ic{float}, \ic{double}). Some of these types require more memory in order to represent larger numbers, while other types use much less memory and can only represent small numbers. \autoref{table:numeric_types} summarizes this tradeoff between memory storage and numeric range or precision.

\begin{table}[h!]
\centering
\begin{tabular}{ |c|c|p{4.5cm}|p{4.5cm}| }
 \hline
 Type & Storage & Minimum Value & Maximum Value \\
 \hline
 \hline
 \ic{byte} & 8 bits & -128 & 127 \\
 \hline
 \ic{short} & 16 bits & -32,768 & 32,767 \\
 \hline
 \ic{int} & 32 bits & -2,147,483,648 & 2,147,483,647 \\
 \hline
 \ic{long} & 64 bits & -9,223,372,036,854,775,808 & 9,223,372,036,854,775,807 \\
 \hline
    \ic{float} & 32 bits & Approximately $-3.4 \times 10^{38}$ with 7 significant digits & Approximately $3.4 \times 10^{38}$ with 7 significant digits \\
 \hline
    \ic{double} & 64 bits & Approximately $-1.7 \times 10^{308} with 15 significant digits$ & Approximately $1.7 \times 10^{308}$ with 15 significant digits \\
 \hline
\end{tabular}
\caption{Java's six primitive numeric data types along with their memory requirements, range, and precision.}
\label{table:numeric_types}
\end{table}

In this course, we will focus on two of these numeric types: \ic{int} for integers and \ic{double} and floating-point numbers. We use \ic{int}s frequently in Java programs because integers, or whole numbers, show up naturally in everyday life. We also use \ic{double}s frequently in Java programs because floating point numbers often show up naturally in scientific applications. A \ic{double} in Java can either be written as a number with a decimal point (e.g. \ic{3.1415} or \ic{4.0}) or using scientific notation (e.g. \ic{6.022e23}). The \ic{e} in this notation is shorthand for ``times 10 to the power of \_\_\_\_''. For example, \ic{6.022e23} is equivalent to $6.022 \times 10^{23}$.

\begin{example}
What are examples of integers in everyday life? \\

\noindent \emph{Answer}: Age (7), number of siblings (2), the result of a dice roll (5), approximate population in New Jersey (8,908,520)
\end{example}

\begin{example}
What are examples of floating point numbers in everyday life? \\

\noindent \emph{Answer}: GPA (4.0), price of a chocolate bar (\$1.99), balance in bank account (\$514.66), average rating out of 5 stars (4.5 out of 5).
\end{example}

\begin{example}
What is the value of \ic{1.23e3}? \\

\noindent \emph{Answer}: It is $1.23 \times 10^3 = 1.23 \times 1000 = 1230$.
\end{example}

\subsection{True/false data}
To represent true/false, yes/no, or on/off data, Java uses \ic{boolean}s. A \ic{boolean} can only be one of two possible values: \ic{true} or \ic{false}. For example, the statement ``The sky is blue right now'' can only either be true or false. If it is daytime, it is most likely true. If it is nighttime or if the sun is currently setting, then it is most likely false. Similarly the statement ``2 plus 2 equals 4'' is clearly true, so we can represent this as a \ic{boolean}. In the same way, we can see that ``3 times 4 is 11'' is false. \ic{boolean}s may seem simple, but we will see soon how to combine them in complex and powerful ways.

\subsection{Textual data}
In general, Java has two main types to represent textual data: a (1) \ic{char} for a single letter or symbol or a (2) \ic{String} for a sequence of multiple letters. A \ic{char} is a single alphanumeric character or symbol, like the ones you can type on your keyboard, that is expressed in Java using single quotes. \ic{'A'}, \ic{'z'}, and \ic{'\$'} are all examples of chars in Java. A \ic{String} is a sequence of characters that is expressed in Java using double quotes. \ic{``Hello World''} is an example of \ic{String}.

\begin{example}
Is \ic{'@'} a valid \ic{char}? \\

\noindent \emph{Answer}: Yes!
\end{example}

\begin{example}
Is \ic{``P''} a valid \ic{char}? \\

\noindent \emph{Answer}: No. A \ic{char} must use single quotes.
\end{example}

\begin{example}
Is \ic{'ABC'} a valid \ic{char}? \\

\noindent \emph{Answer}: No. A \ic{char} must only be a single character.
\end{example}

\begin{example}
What are some examples of \ic{String}s in everyday life? \\

\noindent \emph{Answer}: Text messages, names of friends, email addresses, a poem
\end{example}

\subsection{Summary of five types}
Java has many built-in data types for different types of data: numeric, true/false, and text. In this section, we have focused on five of these types: \ic{int} for integers, \ic{double} for floating point numbers, \ic{boolean} for true/false values, \ic{char} for single characters, and \ic{String} for a sequence of characters. \autoref{table:types_summary} summarizes these five types.

\begin{table}[h!]
\centering
\begin{tabular}{ |c|c|c| }
 \hline
 Type & Description & Examples \\
 \hline
 \hline
 \ic{int} & integers & 5, -100, 1,234,567 \\
 \hline
 \ic{double} & floating point number & 3.75, 6.022e23 \\
 \hline
 \ic{boolean} & true/false value & \ic{true}, \ic{false} \\
 \hline
 \ic{char} & characters & \ic{'A'}, \ic{'\%'}, \ic{'5'} \\
 \hline
 \ic{String} & sequences of characters & \ic{``Hello''}, \ic{``123 Happy St.''} \\
 \hline
\end{tabular}
\caption{A summary of five of Java's built-in types}
\label{table:types_summary}
\end{table}

\section{Operations}
Now that we have learned about data types, we will next learn about the operations we can perform on the the different data types. Specifically, we will learn about arithmetic and comparison operators used with numeric types (e.g. \ic{int}s and \ic{double}s). Then, we will discuss logical operators used with \ic{boolean}s. Finally, we will learn about operations we can perform on \ic{String}s.

\subsection{Arithmetic operators}
Arithmetic operators are generally the ones we are familiar with from elementary school, such as addition (\ic{+}), subtraction (\ic{-}), multiplication (\ic{*}), and division (\ic{/}). For example, \ic{5 + 3} equals \ic{8}, \ic{5.55 - 1.11} equals \ic{4.44}, and \ic{1.5 * 10} equals \ic{15}. Note that just like in regular math, the operator \ic{-} can also refer to negation of the number it's written before. Therefore, \ic{-3} is a correct expression, as well as \ic{5 + -3}, which equals 2.

Of these basic arithmetic operators, division is the trickiest because it behaves in a special way with integers. In Java, if you were to divide two integers, you would get an integer as a result. This is intuitive in some cases. For example, \ic{20 / 5} would equal \ic{4}. However, in other cases, we may lose a fractional component. For instance, \ic{21 / 5} also equals \ic{4}. You might think that \ic{21 / 5} should equal \ic{4.2}, but the fractional piece (everything after the decimal point) gets \emph{truncated}, or dropped completely. This is what we call \emph{integer truncation}. It is worth nothing that truncation is different than \emph{rounding}: as a simple example, rounding would cause 3.9 to become 4, whereas truncating would cause 3.9 to become 3.

\begin{example}
What is \ic{2 / 2}? What about \ic{3 / 2}? What about \ic{4 / 2}? \\

\noindent \emph{Answer}: \ic{2 / 2} equals {1}. \\
\ic{3 / 2} also equals {1} (after dropping the \ic{.5} from the answer \ic{1.5}). \\
\ic{4 / 2} equals {2}.
\end{example}

Java also has an extra arithmetic operator, called the ``modulo'' operator (or ``mod'' for short), that looks like a percent symbol \ic{\%}. Its purpose is to compute the remainder when dividing two numbers. For example, \ic{13 \% 3}, which we pronounce as ``thirteen mod three'', is equal to \ic{1} because 13 divided by 3 gives us 4, remainder 1.

\begin{example}
What is \ic{15 \% 10}? What about \ic{6.25 \% 3}? \\

\noindent \emph{Answer}: \ic{15 \% 10} equals \ic{5}. \\
\ic{6.25 \% 3} equals \ic{0.25}.
\end{example}

In addition to these five arithmetic operators, Java also provides additional mathematical functions that you might find on a basic scientific calculator, such as square root, absolute value, logarithms, etc. For example, \ic{Math.sqrt(25)} equals \ic{5} and \ic{Math.abs(-10)} equals \ic{10}. \autoref{table:math_functions} lists a few examples of mathematical functions provided by Java.

\begin{table}[h!]
\centering
\begin{tabular}{ |c|c|c| }
 \hline
 Function & Description & Example \\
 \hline
 \hline
 \ic{Math.abs(x)} & Absolute value of \ic{x} & \ic{Math.abs(-10)} equals \ic{10} \\
 \hline
 \ic{Math.pow(a, b)} & \ic{a} to the power of \ic{b} ($a^b$) & \ic{Math.pow(2, 3)} equals \ic{8} \\
 \hline
 \ic{Math.sqrt(x)} & Square root of \ic{x} & \ic{Math.sqrt(16)} equals \ic{4} \\
 \hline
\end{tabular}
\caption{A list of a few mathematical functions provided in Java}
\label{table:math_functions}
\end{table}

Of course, we can combine all of these arithmetic operators and mathematical functions in Java to perform complex computation. For example, we might write \ic{Math.sqrt(9) + (5 * 2) - Math.abs(-3)}. Just as you would on a calculator, you can use parentheses to indicate the order of operations in an expression.

\begin{example}
What is the result of \ic{Math.sqrt(9) + (5 * 2) - Math.abs(-3)}? \\

\noindent \emph{Answer}: \ic{Math.sqrt(9) + (5 * 2) - Math.abs(-3)} equals \ic{3 + 10 - 3} which is equal to \ic{10}.
\end{example}

\subsection{Comparison operators}
Comparison operators are operators that compare numeric data. You are probably familiar with the following comparison symbols from math class: $=, <, \leq, >$, and $\geq$. In Java, we use the symbols \ic{==}, \ic{!=}, \ic{<}, \ic{<=}, \ic{>}, \ic{>=}. The result of any comparison operation is a \ic{boolean}. For example, the result of \ic{5 >= 3} is \ic{true}. \autoref{table:comparison_ops} summarizes a list of these comparison operators along with examples.

\begin{table}[h!]
\centering
\begin{tabular}{ |c|c|c|c| }
 \hline
 Operator & Description & Example 1 & Example 2 \\
 \hline
 \hline
 \ic{==} & Equals & \ic{5 == 5} is \ic{true} & \ic{3 == 5} is \ic{false} \\
 \hline
 \ic{!=} & Not equal & \ic{5 != 5} is \ic{false} & \ic{3 != 5} is \ic{true} \\
 \hline
 \ic{<} & Less than & \ic{5 < 5} is \ic{false} & \ic{3 < 5} is \ic{true} \\
 \hline
 \ic{<=} & Less than or equal to & \ic{5 <= 5} is \ic{true} & \ic{3 <= 5} is \ic{true} \\
 \hline
 \ic{>} & Greater than & \ic{5 > 5} is \ic{false} & \ic{3 > 5} is \ic{false} \\
 \hline
 \ic{>=} & Greater than or equal to & \ic{5 >= 5} is \ic{true} & \ic{3 >= 5} is \ic{false} \\
 \hline
\end{tabular}
\caption{A list of the comparison operators, along with examples}
\label{table:comparison_ops}
\end{table}

\subsection{Logical operators}

Logical operators are used to manipulate boolean values. Like comparison operators, the result of an expression with logical operators is a boolean. The three most common logical operators in Java are \ic{\&\&} (logical And), \ic{||} (logical Or) and \ic{!} (logical Not). There are also other operators (|, \&, \^) that we will not cover. Finally, the equality and inequality operators (==, !=) can be applied on boolean values too.

The logical And takes two boolean values and returns true if and only if both values are true. To use it, write the operator between the values. For example, \ic{false \&\& true} applies And to false and true, and the result is false. The following table summarizes the values of And for every input:


\begin{table}[h!]
\centering
\begin{tabular}{ |c|c|c| }
 \hline
 A & B & \ic{And (A \&\& B)} \\
 \hline
 \hline
 false & false & false \\
 \hline
 false & true & false \\
 \hline
 true & false & false \\
 \hline
 true & true & true\\
 \hline
\end{tabular}
\caption{The truth table of And}
\label{table:And}
\end{table}

The logical Or takes two boolean values and returns true when at least one of them is true. Notice that, unlike its usage in language, Or is not exclusive - \ic{A || B} does not mean ``either A or B'' but ``at least one of the two - A, B''. The following table summarizes the values of Or for every input:


\begin{table}[h!]
\centering
\begin{tabular}{ |c|c|c| }
 \hline
 A & B & \ic{Or (A || B)} \\
 \hline
 \hline
 false & false & false \\
 \hline
 false & true & true \\
 \hline
 true & false & true \\
 \hline
 true & true & true\\
 \hline
\end{tabular}
\caption{The truth table of Or}
\label{table:Or}
\end{table}

The logical Not operates on one boolean value, and inverts it. It is written before the value, so for example \ic{!false} is equal to true. 

Like arithmetic and comparison operators, we can combine different logic operators, and even comparison operators, in the same expression.

\begin{example}
What is the result of \ic{(!(2 > 3)) || (0 == 1)}? \\

\noindent \emph{Answer}: \ic{2 > 3} is false. Therefore, \ic{!(2 > 3)} is true. Since at least one of \ic{!(2 > 3)}, \ic{0 == 1} is true, the whole expression is true.
\end{example}

\subsection{Operator Precedence}

We already saw that parentheses can be used to mark the order in which we want to apply operators. What happens, however, when we do not specify the order ourselves? For example, what is the value of \ic{2 > 3 || 0 == 1}? It turns out that in this case, the order of operation the Java language will choose is the same as for the expression \ic{(2 > 3) || (0 == 1)}. This is because some operators take precedence, and are computed before others. In this case, the first operator to be computed is \ic{>}, and it interprets its arguments as the nearest, 2 and 3. Then \ic{==} is executed, interpreting its arguments as 0 and 1. Only then \ic{||} is executed, and the arguments it recognizes are the results of the previous comparisons.

We will not cover the complete order of precedence (that also includes many operators not covered here), and the way ties are broken (for example, in \ic{A || B || C}). Important rules to remember are:

\begin{itemize}
\item Parentheses take precedence over any other operator.
\item Logical And takes precedence over logical Or. 
\item Multiplication takes precedence over addition.
\item Arithmetic operation take precedence over comparisons, which take precedence over logical operators.
\end{itemize}

When in doubt, use parentheses. Even if the result is the same, this can help a reader follow your code more easily. 


\subsection{\ic{String} methods}

Java has a set of operations we can use for building strings. The most common operation we will need is concatenation, or combining two strings. The same sign for the addition operator, +, is used to concatenate strings. For example, \ic{"Hello" + " " + "world"} will result in the string ``Hello world''. Notice that concatenation does not add a space between strings. If we want to concatenate two words, we need to explicitly write a space between them, as in the previous or following example: \ic{"Two " + "words"}.

Another common operation we would like to do with strings is to compare them. For example, check whether two strings are the same. The == operator works on strings as well, but it does not do what we expect. Instead, it checks whether the two strings reference the same object (we will learn about objects later in the course). Instead, when we want to compare two strings, we will use a method called "equals". If A and B represent two strings, we can ask whether they are equal in the following way: \ic{A.equals(B)}. Note that upper and lower case letters are considered different, and that spaces are also counted for comparisons.


\begin{example}
What is the result of \ic{"Room".equals("room")}? \\

\noindent \emph{Answer}: Since the first letter in the two strings differs, the result is false.
\end{example}


\begin{example}
What is the result of \ic{("one" + "two").equals("one two")}? \\

\noindent \emph{Answer}: The result of the concatenation is "onetwo". Since it does not have a space, it is unequal to "one two", and the result is false.
\end{example}

Strings are very common in programming, and so Java has many more string methods. We will learn about them later, as we start to write our own programs.

\section{Variables}
We have now seen data types and operators. So far, we have only worked with \emph{literals}, meaning we have given \emph{literal} examples of values of data types, such as \ic{5} for \ic{int}s or \ic{``Hello World''} for \ic{String}s.

\begin{definition}
A \emph{literal} is an explicit data value used in a program.
\end{definition}

Sometimes, however, we might want to use a \emph{placeholder} to refer to an item of data. This often allows our Java programs to be more general. For example, instead of typing \ic{``John Doe''} literally in our program, we might use \ic{fullname} as a \emph{variable}, or a placeholder, to refer to a person's name on a document.

\begin{definition}
A \emph{variable} is a name for a location in memory used to hold a data value.
\end{definition}

\noindent To create and use a variable, we typically need to do three main things:
\begin{enumerate}
 \item Choose what type of data it will represent (e.g. \ic{int} or \ic{boolean})
 \item Choose what we want to call it (e.g. \ic{name} or \ic{fullname} or \ic{person})
 \item Store a value into the variable (e.g \ic{``John Doe''} or \ic{``Mary Jane''})
\end{enumerate}
We can accomplish these three tasks in two main steps: \emph{declaration} and \emph{assignment}.

\begin{definition}
A \emph{declaration statement} (1) reserves a portion of memory space large enough to hold a particular type of value and (2) indicates the name by which we refer to that location.
\end{definition}

\noindent Examples of declaration statements include:
\begin{itemize}
 \item \ic{String name;}
 \item \ic{int age;}
 \item \ic{double gpa;}
 \item \ic{boolean isOrganDonor;}
\end{itemize}

\begin{definition}
An \emph{assignment statement} sets and/or resets the value stored in the storage location denoted by a variable name; in simpler terms, it stores, or \emph{assigns}, a value to a variable.
\end{definition}

\noindent Examples of assignment statements include:
\begin{itemize}
 \item \ic{name = "John Doe";}
 \item \ic{age = 17;}
 \item \ic{gpa = 3.75;}
 \item \ic{isOrganDonor = true;}
\end{itemize}
	Of course, the values you store into variables must match the type you declared them to be. Otherwise, you will encounter a compiler error. For example, \ic{int x = "x";} would lead to an error.	
In Java, you are allowed to combine a declaration statement and an assignment statement into a single line. For example, instead of a line \ic{String name;} followed by another line \ic{name = "John Doe";}, you can write a single line, \ic{String name = "John Doe";}.	
\begin{example}	
  What will the following code snippet print out?
    Here you should be aware that using ``System.out.print'' will print exactly what it is given and nothing more,
    whereas ``System.out.println'' will add a new line to the end, such that the next thing that gets printed will be on a new line.
  \begin{code}	
    String a = "computer  science ";	
    int b = 4;	
    String c = " ever";	
    System.out.print(a);	
    System.out.print(b);	
    System.out.print(c);	
    System.out.println();	
    c = " life";	
    System.out.print(a);	
    System.out.print(b);	
    System.out.print(c);	
    System.out.println();	
  \end{code}	
\end{example}	
It is important to note the difference between the single equals sign used for assignment and the double equals sign used for comparison. In \ic{age = 17;} the single equals sign represents an \emph{assignment statement}, meaning we are storing the value 17 into a variable named \ic{age}. In contrast, in \ic{age == 17}, the double equals sign represents the fact that we are comparing the value of age to 17, meaning we are asking a yes/no question, whether or not \ic{age} is equal to 17. Look at the following code snippet and make sure you understand the difference between \ic{=} and \ic{==}.	
\begin{code}	
  int a = 5;	
  int b = 3;	
  System.out.println(a == b);   // prints false	
  a = b;                        // now a is 3	
  System.out.println(a == b);   // prints true	
\end{code}	
When using the \ic{=} symbol to assign to a variable, the variable you are assigning to should always go on the left. For example, we can write \ic{x = 5;} to assign \ic{5} to \ic{x}, but \ic{5 = x;} is not valid Java code, since we can't assign to a literal. If we have two different variables, say \ic{x} and \ic{y}, then \ic{x = y;} and \ic{y = x;} are both valid lines of code, but they do different things. If we write \ic{x = y;}, then the value of \ic{x} changes to become the value of \ic{y}. If we write \ic{y = x;}, then the value of \ic{y} changes to become the value of \ic{x}.	
\begin{example}	
  Fill in some code in the middle of the following snippet to swap the values of \texttt{a} and \texttt{b}. The code should print \ic{ABBA} if you do this correctly. \emph{Hint:} you may need to define a third variable!	
  \begin{code}	
    String a = "A";	
    String b = "B";	
    System.out.print(a);	
    System.out.print(b);	
    // Your code here	
    // Swap the values of a and b	
    System.out.print(a);	
    System.out.println(b);	
  \end{code}	
\end{example}	
A declaration statement will also lead to a compiler error if the variable has already been declared. For example, if we have a line \ic{int x = 5;} followed by another line \ic{int x = 7;}, the compiler will see the second line and complain that the variable \ic{x} has already been declared. You might imagine that this could cause some difficulties: in a large program, we might want to use the variable \ic{x} to mean different things at different points, without worrying about whether \ic{x} has been declared earlier. Luckily, Java and most other programming languages provide a tool to manage this, called \emph{scope}.	
\begin{definition}	
  The \emph{scope} of a variable is the section of code where it can be referenced.	
\end{definition}	
In all of the examples we have seen so far, the scope of a variable consists of all lines after the declaration statement for that variable. In later lectures, we will see how more advanced tools can restrict the scope of variables.	
Previously we saw that we can combine multiple operations with literals to perform more complex computation. In the same way, we can combine literals and variables and operators altogether.	
\begin{definition}	
An \emph{expression} is a combination of one or more operators and operands that usually performs a calculation.	
\end{definition}	
For example, look at the following code snippet for computing the area of a circle.	
\begin{code}	
  double radius = 4.5;	
  double area = 3.14*radius*radius;	
  System.out.print("Radius: ");	
  System.out.println(radius);	
  System.out.print("Area: ");	
  System.out.println(area);	
\end{code}	
We can even update the value of a variable with an expression that includes that same variable. For example, \ic{x = x + 5;} will increase the value of \ic{x} by 5.	
\begin{example}	
  To convert from Celsius to Fahrenheit, multiply by 1.8 and then add 32. Fill in a single line of code below to convert \ic{temp} from Celsius to Fahrenheit;	
  \begin{code}	
    double temp = 12;	
    System.out.print("The temperature is ");	
    System.out.print(temp);	
    System.out.println(" degrees Celsius");	
    // Your code here	
    System.out.print("The temperature is ");	
    System.out.print(temp);	
    System.out.println(" degrees Fahrenheit");	
  \end{code}	
\end{example}	
There are a few shorthand notations for updating variables. If you want to act on a variable \ic{var} with a single operator, such as \ic{var = var - 7;} or \ic{var = var * 2.3;}, you can abbreviate this as \ic{var -= 7;} or \ic{var *= 2.3} respectively. This works for any of the arithmetic operators: you can use \ic{+=}, \ic{-=}, \ic{*=}, \ic{/=}, or \ic{\%=}.	
When mixing variables in expressions, it is important to be aware of their types and to know how variables of different types combine together. When working with numeric types, if you combine two values of the same type with an operator, the result will also be of that type. A common pitfall for this rule is division: if you divide two \ic{int}s, the result will be an \ic{int}.	
\begin{code}	
  int x = 8;	
  int y = 5;	
  System.out.println(x/y); // prints 1	
\end{code}	
Even though 8/5 = 1.6, Java will interpret \ic{x/y} as another \ic{int}, so it becomes \ic{1}. This is just like dividing integer literals, but with variables, you have to keep in mind what the variable types are so you will know if division will be truncated.	
When you combine two different numeric types, the result will be a value of the more precise type. The most common example of this is combining an \ic{int} with a \ic{double}, which will result in a \ic{double}. This is called "automatic type conversion".
You can also override Java's default rules for assigning types to expressions by using \emph{casting}. Casting allows you to tell Java to interpret an expression as a given type by putting the desired type in front of the expression in parentheses: for example, \ic{(int) x} casts \ic{x} to an \ic{int}. For now, we will only use casting to convert between different numeric types. Later on we will learn more detailed rules for casting.	
\begin{example}	
  Look at the following examples and make sure you understand how Java assigns a type to each expression. Keep track of the parentheses so you know what is being casted.	
  \begin{code}	
    double eight = 8.0;	
    int five = 5;	
    System.out.println(eight/five); // prints 1.6	
    System.out.println((int)(eight/five)); // prints 1	
    System.out.println(eight/((double)five)); // prints 1.6	
    System.out.println(((int) eight)/five); // prints 1	
    System.out.println(((int) eight)/((double) five)); // prints 1.6	
  \end{code}	
\end{example}	

As previously mentioned, Java will attempt ``automatic type conversion" in case you try to assign a lower-precision value to a type that has higher precision without casting yourself. Java will throw an error, however, if you try to assign a higher-precision value to a lower-precision type. This can get very confusing. So it's best to avoid type conversion in your code when you can, and explicitly cast the type when you must. 

\begin{example}
  What will happen if you compile and run this code? 
  \begin{code}
    int intEight = 8;
   double doubleEight = intEight;
   System.out.println(doubleEight);
  \end{code}
  
  How about this code? 
  \begin{code}
  double doubleEight = 8.0;
  int intEight = doubleEight; 
  System.out.println(intEight);
  \end{code}
  
\noindent \emph{Answer}: In the first example, Java will happily use automatic type conversion to convert the lower-precision int (intEight) to the higher-precision double (doubleEight). The output of the code is \ic{8.0}, the double version of \ic{8}. In the second example, however, Java cannot use automatic type conversion to go from a higher-precision double (doubleEight) to a lower-precision int (intEight). Java will throw an error when you try to compile this code. 
\end{example}

\subsection{Variable names}

There are some rules variable names need to follow (with some exceptions omitted):

\begin{itemize}
\item A variable name must begin with a letter.
\item The digits 0--9 and the character \_ can appear in a variable name, as long as it does not begins with them. 	
\item Spaces and other special characters (like \%, \#, and \textasciicircum) cannot appear in a variable name.	
\item A keyword used for the Java language itself cannot be used as a variable name (for example, \ic{String}).	
\end{itemize}	
Lower- and upper-case letters are considered different in variable names, and so \ic{name} and \ic{Name} are considered different names. There are naming conventions for variables, that help us read code easily and understand what the function of each variable is:	
\begin{itemize}	
\item Variables should have short, meaningful names. Limit variable names to a few short words, and make sure that a reader unfamiliar with the code could easily understand what the purpose of each variable is. 	
\item Most variable names should start with a lower-case letter, and each new word after the first should start with an upper-case letter (for example, \ic{firstName}). This naming pattern is called camel case, and is used for many other languages.	
\item Constant variables, whose value should not change, should be written with all caps (upper-case letters), and different words should be separated by an underscore (for example, \ic{MAXIMAL\_WEIGHT}).	
\item Temporary variables, that are only used for a short part of the code, can have single letter names. Typically, \ic{i}, \ic{j}, \ic{k}, \ic{m} and \ic{n} are used for integer values, and \ic{c}, \ic{d} and \ic{e} for characters.
\end{itemize}

Programming is often a collaborative process, in which different people work on the same body of code. This makes good naming conventions, and good code writing habits in general, an extremely important part of programming. In fact, you might find yourself going back to code you have written a while ago, and find that you are unfamiliar with it.

\ja{Add list of Java reserved key words in Appendix}

\ja{Make a decision on whether we want to cover non-int and double at all.}

\exercisesection

\begin{exercise}
Complete the blanks in the following variable declarations so that the variable type matches the value assigned to it. If multiple types can hold it, write the smallest type.
\begin{enumerate}
\item \ic{\underline{double} temperature = 98.6;}
\item \ic{\underline{\hspace{2cm}} isReal = false;}
\item \ic{\underline{\hspace{2cm}} prefix = 'q';}
\item \ic{\underline{\hspace{2cm}} ZERO = 0;}
\item \ic{\underline{\hspace{2cm}} cap = 200;}
\item \ic{\underline{\hspace{2cm}} velocity = 1.1e100;}
\item \ic{\underline{\hspace{2cm}} mass = 1e5;}
\end{enumerate}
\end{exercise}

\begin{exercise}
Which of the following lines is valid Java code? Assume that no variable has been previously defined. If a line is incorrect, explain which rules are violated in it. Remember that naming conventions are not rules of the langauge, but (important) recommentations.
\begin{enumerate}
\item \ic{String Songname;}
\item \ic{String first paragraph;}
\item \ic{double numerator = 20;}
\item \ic{int STREET\_NAME;}
\item \ic{double index;}
\item \ic{String message = "String";}
\item \ic{String firstLetter = "c";}
\item \ic{int MAX\_VALUE = "10";}
\end{enumerate}
\end{exercise}

\begin{exercise}
Which of the following expressions is a valid Java code? If an expression is valid, determine the result of it. Otherwise, explain why it is not valid.
\begin{enumerate}
\item \ic{1 * 2 * 3}
\item \ic{(1 * 2) * 3}
\item \ic{1 < 2 < 3}
\item \ic{2 - 1 + 3}
\item \ic{2 - (1 + 3)}
\item \ic{(2 - 1) + 3}
\item \ic{true || 2 == 3}
\item \ic{(true || 2) == 3}
\item \ic{3 / 2 + 1}
\item \ic{3 / (2 + 1)}
\item \ic{true \&\& false \&\& true}
\item \ic{!((true == false) \&\& !(2 > 3))}
\end{enumerate}
\end{exercise}

\begin{exercise}
Determine the value of the variable \ic{test}. 
\begin{enumerate}
\item \ic{int test = 4/2;}
\item \ic{int test = 3/2;}
\item \ic{double test =4.0/2}
\item \ic{double test =3.0/2}
\item \ic{double test = 3/2.0;}
\end{enumerate}
\end{exercise}

\begin{exercise}
Which of the following expressions is a valid Java code? If an expression is valid, determine the result of it. Otherwise, explain why it is not valid.
\begin{enumerate}
\item \ic{int tmp = 1; int test = (int) tmp;}
\item \ic{int tmp = 1; double test = (double) tmp;}
\item \ic{double tmp = 1.1; int test = (int) tmp;}
\item \ic{double tmp = 1.9; int test = (int) tmp;}
\item \ic{int tmp = 1.1; double test = (double) tmp;}
\item \ic{double tmp = 1.1; short test = (short) tmp;}
\item \ic{int tmp = 3; double test = tmp;}
\item \ic{double tmp = 3; int test = tmp;}
\end{enumerate}
\end{exercise}

\begin{exercise}
Describe ``automatic type conversion" in your own words. 
\end{exercise}

\referencessection

Computer Science: An Interdisciplinary Approach, Robert Sedgewick and Kevin Wayne.

Lewis, John, Peter DePasquale, and Joseph Chase. Java Foundations: Introduction to Program Design and Data Structures. Addison-Wesley Publishing Company, 2010.

Wikipedia

\ab{How do I reference https://www.geeksforgeeks.org/java-naming-conventions? (and should I?)}
