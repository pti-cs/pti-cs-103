\chapter{Input and Output}

Let's look at the ``Hello World'' program that we saw earlier:
\begin{code}
class HelloWorld {

    public static void main(String[] args) {
        System.out.println("Hello World!");
    }

}
\end{code}

When you ran this program, it produced the \emph{output} \ic{Hello World!} on a single line.
Using \ic{System.out.println} is one way in which you can have your programs
produce output. In particular, here \ic{System.out} refers to the \emph{output stream}
that, in our case, is where you see output in the ``Interactions'' tab in DrJava.
The \ic{println} part refers to printing (i.e., outputting) the argument
--the String provided between the parentheses--followed by a linebreak (the \ic{ln} part).
As you might imagine, there are ways to produce output in Java, ranging from printing
to \ic{System.out} with and without linebreaks to writing out image files that
plot data. We can also provide \emph{inputs} to a program in order
to affect what is computed and ultimately outputted.

In this chapter, we will be exploring different ways to produce program output
as well as take program inputs.

\section{Output}
The output of a program is what lets the user who runs it
see any results of the work done in the program.
Consider the following program:
\begin{code}
class HelloWorld {

    public static void main(String[] args) {
        String hi = "Hello";
        String earth = "World";
        String result = hi + " " + earth;
    }

}
\end{code}
This program has no output, so although it will compile and run, you won't
be able to tell that it has done anything!
If you added an output statement, then you'd be able to see that
the program does something. We can try using \ic{System.out.println},
which we've seen already.
\begin{code}
class HelloWorld {

    public static void main(String[] args) {
        String hi = "Hello";
        String earth = "World";
        String result = hi + " " + earth;
        System.out.println(result + "!");
    }

}
\end{code}
Now this program will produce the output as seen when we first ran Hello World program:
\begin{code}
> run HelloWorld
Hello World!
>
\end{code}

Let's take a look at some other ways of printing.
Instead of \ic{System.out.println}, we can use \ic{System.out.print}, which
doesn't print the linebreak after the String provided.
If you change \ic{System.out.println} to \ic{System.out.print} above, and hit
``Run'' in DrJava, you'll see something like this:
\begin{code}
> run HelloWorld
Hello World!>
\end{code}
The \ic{>} that indicates that you can type input into the ``Interactions''
tab is now on the same line as \ic{Hello World!} because the linebreak
wasn't output.

\ic{System.out.println} actually does not require a String to be provided,
so we can actually get the linebreak back by adding a
\ic{System.out.println} statement without the String provided:
\begin{code}
class HelloWorld {

    public static void main(String[] args) {
        String hi = "Hello";
        String earth = "World";
        String result = hi + " " + earth;
        System.out.print(result + "!");
        System.out.println();
    }

}
\end{code}
We now have output that looks like this again:
\begin{code}
> run HelloWorld
Hello World!
>
\end{code}

An alternative is to use the \emph{newline} character \ic{\textbackslash n} that represents a
linebreak:
\begin{code}
class HelloWorld {

    public static void main(String[] args) {
        String hi = "Hello";
        String earth = "World";
        String result = hi + " " + earth;
        System.out.print(result + "!\n");
    }

}
\end{code}

We can provide non-String arguments to \ic{System.out.print} and \ic{System.out.println} as well:
\begin{code}
class HelloWorld {

    public static void main(String[] args) {
        String hi = "Hello";
        String earth = "World";
        String result = hi + " " + earth;
        System.out.print(result + "!");
        System.out.println();
        System.out.println(1);
        System.out.println(10/3.0);
    }

}
\end{code}
Try compiling and running the program above.
Note that \ic{System.out.println(10/3.0);} ends up printing
\ic{3.3333333333333335}, with 16 digits after the decimal point.

We can specify how many digits after the decimal point as well
as specifying a lot of other interesting formatting options
using \ic{System.out.printf} (the \ic{f} here refers to formatting).

When using \ic{System.out.printf}, we provide several arguments
in between the parentheses, with the arguments being separated by
commas.
For example, we can replace \ic{System.out.println(10/3.0)} above
with \ic{System.out.printf("\%f\textbackslash n", 10/3.0)}.
The first argument \ic{"\%f\textbackslash n"} is a String that describes the format of the output.
In this case, \ic{\%f} refers to a floating-point number (i.g. a \ic{float} or \ic{double}),
and by default will print 6 places after the decimal point.
The \ic{\textbackslash n} part is the newline character.
Because we specified \ic{\%f} in the formatting String, we must also provide
a floating point number to replace the \ic{\%f} in the output. Here,
we have provided \ic{10/3.0}.
If we perform this replacement, the output looks as follows:
\begin{code}
> run HelloWorld
Hello World!
1
3.333333
> 
\end{code}

We can specify only 3 decimal places by changing \ic{\%f} to \ic{\%.3f}.
Note that the number is \emph{rounded}, not truncated, so if we have a value like
3.3335 and format it with \ic{\%.3f}, it will be printed out as \ic{3.334}.
The following table provides some common format specifiers (we have already seen
\ic{\%f}):

\begin{tabular}{|l|l|}
\hline
Format Specifier & Effect\\
\hline
\%s,\%S & Formats String\\
\%f & Formats floating point (precision provided between \% and f)\\
\%d & Formats integer\\
\%c & Formats character\\
\%b, \%B & Formats Boolean\\
\hline
\end{tabular}

Using this table, we can actually get the same input that we've been getting
with only one print statement:
\begin{code}
class HelloWorld {

    public static void main(String[] args) {
        String hi = "Hello";
        String earth = "World";
        String result = hi + " " + earth;
        System.out.printf("%s!\n%d\n%.3f\n", result, 1, 10/3.0);
    }

}
\end{code}
Look at the format String \ic{"\%s!\textbackslash n\%d\textbackslash n\%.3f\textbackslash n"}. There are three format
specifiers: \ic{\%s}, \ic{\%d}, and \ic{\%.3f} that are respectively
used to format the other arguments \ic{result}, \ic{1}, and \ic{10/3.0}.
The format String also includes formatting around these arguments, such
as the exclamation point after \ic{result} and the newline characters.
 
\begin{example}
Consider the following program:
\begin{code}
class Mystery {

    public static void main(String[] args) {
        System.out.printf("%.3f\n%.4f%d\n", 1.252525, 1.353535, 3);
    }

}
\end{code}
What is the output of the program?
\noindent \emph{Answer}:
\begin{code}
1.253
1.35353
\end{code}
\end{example}

\section{Input}
So far, each the program that we've seen produces the same output no
matter how many times it is run. If we want to change the output,
we have to edit the program, recompile it, and run it again.
In practice, the user--the person who runs the program--will
not be able to edit and recompile the program.
We can enable a user to affect the result of the program
by taking user \emph{input}.

In order to take user input, we will be using the \ic{Scanner} class
along with the \ic{System} class
rather than just the \ic{System} class.
Don't worry much about what classes are now; we'll cover
them in depth later.

A brief explanation follows for the interested:
Classes can contain \emph{fields} and \emph{methods}.
Fields are variables contained in a class and are used
for storing data. The \ic{out} in \ic{System.out} is
a field in the \ic{System} class
that refers to where the output should go.
Methods (which we will explain in more detail later) are named bits of code
that are parameterizable by the provided arguments. These
arguments are provided in between parentheses.
For example, \ic{System.out.printf} above is a method.
Fields and methods may be \ic{static}; static fields and methods
of a class can be accessed by using the class name followed
by a period and then the name of the field. For example,
\ic{out} is a static field of \ic{System} and 
so can be accessed by \ic{System.out}.
Another example is that \ic{pow} is a static method
of \ic{Math} and so can be accessed by \ic{Math.pow}.
Non-static fields and methods must be accessed
through an \ic{instance} of a class. For example,
\ic{out} is an instance of another class.
Since \ic{printf} and \ic{println} are non-static methods
of that class, they are accessed
by referring to the instance \ic{out} as in
\ic{System.out.printf} or \ic{System.out.println}.
We will need to make an \emph{instance} of the
\ic{Scanner} class in order to take user input.

To use the \ic{Scanner} class, we must first import it
with the following statement:
\begin{code}
import java.util.Scanner;
\end{code}
The import statement should go \emph{before} the class declaration in your file, so
if we were to add the import statement to the last ``Hello World'' program modification
we've seen, it would look like this:

\begin{code}
import java.util.Scanner;

class HelloWorld {

    public static void main(String[] args) {
        String hi = "Hello";
        String earth = "World";
        String result = hi + " " + earth;
        System.out.printf("%s!\n%d\n%.3f\n", result, 1, 10/3.0);
    }

}
\end{code}

In order to use the \ic{Scanner} class, we must first \emph{create an instance} of it.
We will cover what this means later, when we talk about classes in more detail.
An instance of \ic{Scanner} named \ic{input} is created as follows:
\begin{code}
Scanner input = new Scanner(System.in);
\end{code}
Here, \ic{System.in} specifies that we want to take user input
on the command line. (If you use DrJava, user input will be taken
in the ``Interactions'' tab.) Each user input is followed by an enter/return.

Here is an example program that takes user input:
\begin{code}
import java.util.Scanner;

class HelloPlanet {

  public static void main(String[] args) {
    Scanner input = new Scanner(System.in);
    System.out.print("Enter a planet name: ");
    String planet = input.next(); // get next user input as a String
    System.out.printf("Hello %s\n", planet);
    input.close(); // close the Scanner
  }

}
\end{code}

Try running it and see what happens!
Here we can see that \ic{input.next()} can be used to get the next
user input as a \ic{String}. We can get user input of other types as well,
which we will see briefly. Note that when we are done using a
\ic{Scanner}, we close it. In this case, since our \ic{Scanner}
instance is called \ic{input}, we close it with \ic{input.close()}.
We cannot use \ic{input} to take input any more after it has been closed.
Let's give it a try. Try running the following:
\begin{code}
import java.util.Scanner;

class HelloPlanet {

  public static void main(String[] args) {
    Scanner input = new Scanner(System.in);
    input.close(); // close the Scanner too early
    System.out.print("Enter a planet name: ");
    String planet = input.next(); // get next user input as a String
    System.out.printf("Hello %s\n", planet);
  }

}
\end{code}
You should get an error output: \ic{java.lang.IllegalStateException: Scanner closed}.

Let's now look at how to take other kinds of input:
\begin{code}
import java.util.Scanner;

class Adder {

  public static void main(String[] args) {
    Scanner input = new Scanner(System.in);
    System.out.print("Enter an integer: ");
    int first = input.nextInt(); // get next user input as an int
    System.out.print("Enter an integer: ");
    int second = input.nextInt(); // get next user input as an int
    System.out.printf("The sum is %d.\n", first + second);
    input.close();
  }

}
\end{code}
Try running the code!
If you provide values that are integers, then it should sum these
values and provide you with the result. An example of how this might look
follows:
\begin{code}
> run Adder
Enter an integer: -1
Enter an integer: 2
The sum is 1
> 
\end{code}

If you try providing values that are not integers,
then you will get an error:\\\ic{java.util.InputMismatchException}.
The method \ic{nextInt} expects to see an integer value
and cannot handle values of other types.

In addition to \ic{nextInt()} which provides ways of
getting \ic{int}s from user input, there is
also \ic{nextDouble()} which gives a way of getting
\ic{double}s from user input.

\begin{example}
Consider the following program:
\begin{code}
import java.util.Scanner;

class Mystery {

  public static void main(String[] args) {
    Scanner input = new Scanner(System.in);
    System.out.print("Enter a number: ");
    double first = input.nextDouble();
    System.out.print("Enter an number: ");
    double second = input.nextDouble();
    System.out.print("Enter an number: ");
    double third = input.nextDouble();
    System.out.printf("Result:%.3f\n", first + second - third);
    input.close();
  }

}
\end{code}
\begin{itemize}
\item
What is the output of the program when the user gives inputs \ic{1}, \ic{3.5}, \ic{1.5}?
\item
What is the output of the program when the user gives inputs \ic{2.5555}, \ic{3.4444}, and \ic{1.1111}?
\end{itemize}
\noindent \emph{Answer}:
\begin{itemize}
\item
The ouput of the program when the user gives inputs \ic{1}, \ic{3.5}, \ic{1.5}
is \ic{3.000} (followed by a newline).
\item
The ouput of the program when the user gives inputs \ic{2.5555}, \ic{3.4444}, \ic{1.1111}
is \ic{4.889} (followed by a newline).
\end{itemize}
\end{example}
% not covering file I/O
%\section{File I/O}
%We've now seen a little bit of how to output results of a program
%and how to take in user input.
%All of this input and output (I/O) can be seen in text in the ``Interactions'' tab
%of DrJava. There are, however, other ways of handling input.
%In particular, we can use the content of files on the computer as input and
%also output content to files on the computer.
%Let's explore this a little bit here.
%
%\begin{code}
%import java.io.FileReader;
%import java.io.FileWriter;
%import java.io.IOException;
%
%class Adder {
%  public static void main(String[] args) throws IOException {
%    FileReader input = new FileReader("numbers.txt");
%    FileWriter output = new FileWriter("sums.txt");
%    int first = input.read();
%    int second = input.read();
%    int third = input.read();
%    int fourth = input.read();
%    int fifth = input.read();
%    output.write("Sum is " + first + second + third + fourth + fifth);
%  }
%}
%\end{code}
%
%\begin{code}
%import java.io.FileReader;
%import java.io.FileWriter;
%import java.io.IOException;
%
%class Adder {
%
%  public static void main(String[] args) throws IOException {
%    FileReader input = new FileReader("numbers.txt"); 
%    FileWriter output = new FileWriter("sums.txt");
%    int first = input.read();
%    int second = input.read();
%    int third = input.read();
%    int fourth = input.read();
%    int fifth = input.read();
%    output.write("Sum is " + first + second + third + fourth + fifth);
%  }
%
%}
%\end{code}
%
%\begin{code}
%import java.io.FileReader;
%import java.io.PrintWriter;
%import java.io.IOException;
%
%class Adder {
%
%  public static void main(String[] args) throws IOException {
%    FileReader input = new FileReader("numbers.txt");
%    PrintWriter output = new PrintWriter("sums.txt");
%    int first = input.read();
%    int second = input.read();
%    int third = input.read();
%    int fourth = input.read();
%    int fifth = input.read();
%    output.printf("First:%d\n", first);
%    output.printf("First two:%d\n", first + second);
%    output.printf("First three:%d\n", first + second + third);
%    output.printf("First four:%d\n", first + second + third + fourth);
%    output.printf("First five:%d\n", first + second + third + fourth + fifth);
%  }
%
%}
%\end{code}
%
%TODO: introduce \ic{BufferedReader}
%
%\begin{code}
%import java.io.BufferedReader;
%import java.io.FileReader;
%import java.io.PrintWriter;
%import java.io.IOException;
%import java.util.Scanner;
%
%class Adder {
%
%  public static void main(String[] args) throws IOException {
%    Scanner input = new Scanner(new BufferedReader(new FileReader("numbers.txt")));
%    PrintWriter output = new PrintWriter("sums.txt");
%    int first = input.nextInt();
%    int second = input.nextInt();
%    int third = input.nextInt();
%    int fourth = input.nextInt();
%    int fifth = input.nextInt();
%    output.printf("First:%d\n", first);
%    output.printf("First two:%d\n", first + second);
%    output.printf("First three:%d\n", first + second + third);
%    output.printf("First four:%d\n", first + second + third + fourth);
%    output.printf("First five:%d\n", first + second + third + fourth + fifth);
%  }
%
%}
%\end{code}

\exercisesection

\begin{exercise}
Give three different programs that all have the same output as the “Hello World!” program.
\end{exercise}

\begin{exercise}
Write a program that takes a user’s input and repeats it three times in a row back to them on a single line. For example, if a user provides “Hello” as input, your program should print out “HelloHelloHello” and then print a newline character. Use only System.out.print for printing output (i.e., do not use System.out.println nor System.out.printf).
\end{exercise}

\begin{exercise}
Write a program that takes two inputs from the user and repeats them back to the user in the opposite order. For example, if a user provides “foo” and “bar” as input, your program should print out “barfoo” and then print a newline character. Use only System.out.println for printing output (i.e., do not use System.out.print nor System.out.printf).
\end{exercise}

\begin{exercise}
Write a program that takes a user-inputted x value and prints out the y value of a line with slope 2.4 and y-intercept of 3.5. The precision of the result should be to 2 decimal points. Note that the y value can be computed as follows, where x is the user input: \ic{y = 2.4 * x + 3.5}
\end{exercise}
