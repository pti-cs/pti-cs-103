\chapter{Methods}

% Abstraction, modularity, code-reuse, syntax, arguments and calling

Sometimes it is useful to abstract certain computations into methods.
In this chapter, we will be looking at how and why we can use methods to
achieve more concise and easily-maintainable code.

We will first look at some code that does not use methods and identify some
problems with this code. We will then, in Section \ref{sec:methods},
introduce what a method is, including how to define and use them.
In Sections \ref{sec:abstraction} and \ref{sec:modularity}, we will
then talk about how we can use methods to improve the problems that
we noticed with our code. In particular, in Section \ref{sec:abstraction},
we talk about how methods allow us to abstract some of our code to get
more concise and maintainable code, and in Section \ref{sec:modularity},
we talk about how methods allow us to divide our code up into chunks
that we can consider in isolation, allowing us to manage and maintain
our code more easily.

Let us consider the following problem:

We have a class of nine students, split up into three groups of three.
For each group, we want to report the maximum score for each group.
%as well as the maximum score for the students overall.
We will use the variable names \ic{group1s1}, \ic{group1s2}, and \ic{group1s3}
to refer to the three scores achieved by the students in Group 1, and we will use
a similar naming convention for the variable names for the scores achieved by
students in Group 2 and Group 3.

The following snippet of code computes the maximum score in Group 1, storing
the result in \ic{group1Max}.
\begin{code}
int group1Max = group1s1;
if (group1s2 > group1Max) {
  group1Max = group1s2; 
}
if (group1s3 > group1Max) {
  group1Max = group1s3; 
}
\end{code}

If we now wish to extend this to compute the maximum score for each group,
we can copy the code and rename the variables to achieve the following
code:
\begin{code}
int group1Max = group1s1;
if (group1s2 > group1Max) {
  group1Max = group1s2; 
}
if (group1s3 > group1Max) {
  group1Max = group1s3; 
}

int group2Max = group2s1;
if (group2s2 > group1Max) {
  group2Max = group2s2; 
}
if (group2s3 > group2Max) {
  group2Max = group2s3; 
}

int group3Max = group3s1;
if (group3s2 > group3Max) {
  group3Max = group3s2; 
}
if (group3s3 > group3Max) {
  group3Max = group3s3; 
}
\end{code}

\noindent There are a couple of things about the process of getting this resulting code
that aren't particularly ideal.
The first is that it is very easy to make a mistake when copying the code and renaming
the variables. For example, we might have missed renaming \ic{group1Max} to \ic{group3Max}
in the computation of the maximum score for Group 3, leading us to compute an incorrect maximum
value in some cases.

The second issue is that there is a lot of redundancy.
If we look at the code, we can notice that there seem to be a lot of redundant parts
across the code for the three different groups.
If we had ended up with the wrong computation initially and needed to fix it
later, then we would have to make sure we fixed it everywhere, leading to repeated work and the
possibility that we didn't fix one of the copies. For example, if we initially started out with
\ic{group1s3 < group1Max} instead of \ic{group1s3 > group1Max}, and then copied and renamed variables
to compute the maximum scores for the other groups, we would end up with the code below:

\begin{code}
int group1Max = group1s1;
if (group1s2 > group1Max) {
  group1Max = group1s2;
}
if (group1s3 < group1Max) {
  group1Max = group1s3;
}

int group2Max = group2s1;
if (group2s2 > group1Max) {
  group2Max = group2s2;
}
if (group2s3 < group2Max) {
  group2Max = group2s3;
}

int group3Max = group3s1;
if (group3s2 > group3Max) {
  group3Max = group3s2;
}
if (group3s3 < group3Max) {
  group3Max = group3s3;
}
\end{code}

To fix our code, we would not only have to replace \ic{group1s3 < group1Max} with
\ic{group1s3 > group1Max}, but we would also need to replace \ic{group2s3 < group2Max} with
\ic{group2s3 > group2Max} and \ic{group3s3 < group3Max} with \ic{group3s3 < group3Max}.

\section{Methods}\label{sec:methods}
In Java, we can use abstractions in our code in the form of \emph{methods}.
Methods contain bits of code that may compute things using \emph{arguments}
that are passed into the method.

An example of the syntax for a \emph{method definition} is given below:
\begin{code}
public bool negate(bool arg) {
  return !arg; // method body
}
\end{code}
We can identify the different components of the method definition:
\begin{itemize}
\item  The \emph{access modifier} of the method is \ic{public},
\item the \ic{bool} before \ic{negate} gives the \emph{return type} of the method
\item \ic{negate} is the \emph{name of the method},
\item \ic{arg} is the \emph{name of the first argument/parameter} of the method,
\item the \ic{bool} before \ic{arg} it is the \emph{type} of the \ic{arg} parameter,
\item and the code between the curly braces is the method \emph{body}.
\end{itemize}
Inside the method body, there is code that may use the arguments of the method.
In the example above, the body only consists of the \ic{return} statement.
The \ic{return} statement is followed by something of the return type of the method.
This expression, after being evaluated, gives the \emph{return value} of the method.
As soon as a \ic{return} statement is encountered, a method finishes its execution.
\emph{Access modifiers} will be described later on when we talk about Classes,
but note that if an access modifier is omitted, it the method has the \emph{default} modifier.

Other important terms to note are as follows:
The \emph{parameter list} of a method consists of the types of its parameters in the
order they are given in the method definition.
For the above method, it is simply \ic{bool}.
The \emph{method signature} consists of the method name and parameter list.
For the above method, it is \ic{negate(bool)}.
A method is identified by its method signature.

\begin{example}
Consider the following method:
\begin{code}
int squareSum(int arg0, int arg1) {
  int res = arg0 + arg1;
  return res * res;
}
\end{code}
Give the following for the method:
\begin{itemize}
\item Access modifier
\item Return type
\item Arguments
\item Method signature
\end{itemize}

\noindent \emph{Answer}: 
\begin{itemize}
\item Access modifier: default
\item Return type: \ic{int}
\item Arguments: \ic{arg0}, \ic{arg1}
\item Method signature: \ic{squareSum(int, int)}
\end{itemize}
\end{example}

\begin{example}
Consider the following method:
\begin{code}
private void mystery (int age, String name) {
  if (age >= 18) {
    System.out.println(name);
  }
}
\end{code} 
Give the following for the method:
\begin{itemize}
\item Access modifier
\item Return type
\item Name
\item Parameter list
\end{itemize}

\noindent \emph{Answer:}
\begin{itemize}
\item Access modifier: \ic{private}
\item Return type: \ic{void}
\item Name: \ic{mystery}
\item Parameter list: \ic{int, String}
\end{itemize}

\end{example}

We may \emph{call} or \emph{invoke} the above \ic{negate} method from elsewhere in the code by,
for example, using \ic{negate(true)} if we want to call \ic{negate} with argument
\ic{true}.
We must specify the name of the method that we wish to call, followed by the arguments
that we wish to call it on, separated by commas if there are more than one.
The arguments provided in the call must correspond to the types of the arguments
specified in the method definition: if the method definition has the type \ic{bool}
for its first argument, then the first argument supplied in the call should also be
of type \ic{bool}, and if the definition has the type \ic{int} for its second argument,
then the first argument in the call should also be of type \ic{int}, and so on.
A method invocation evaluates to the return value it computes, where this
computation happens with the actual arguments
given in the call substituted for the arguments specified in the definition, so in
this case, the invocation \ic{id(true)} evaluates to \ic{false}.
A method invocation has the same type as its return type, which in this case, is \ic{bool}.

So, for example, we can do something like in following code snippet:
\begin{code}
  bool f = negate(true);
  bool t = negate(negate(true));
\end{code}
After executing the above code, the variable \ic{f} contains the value \ic{false}
and the variable \ic{t} contains the value \ic{true}.

Note that we can also have methods that do not return anything. Such
methods have \ic{void} return types. They may contain a \ic{return}
statement with no expression following \ic{return}.
Calls to these methods cannot be used inside of other expressions.
Two examples of methods with \ic{void} return types are below:
\begin{code}
public void printSum(double a, double b) {
  System.out.println(a + b);
}
\end{code}

\begin{code}
public void printBigger(double a, double b) {
  if (a > b) {
    System.out.println(a);
    return;
  }
  System.out.println(b);
}
\end{code}
Note that the latter example only ever prints the value of either \ic{a} or \ic{b} but
never both because
the \ic{return} statement finishes the execution of the method.

Some methods also do not take any arguments at all. One such example is below:
\begin{code}
public int constantOne() {
  return 1;
}
\end{code}
\noindent This method can be called using \ic{constantOne()}.

\begin{example}
What value does \ic{res} have after the following code snippet is executed?
\begin{code}
int res = constantOne() + constantOne();
\end{code}
\emph{Answer}: The value of \ic{res} is 2.
\end{example}.

Note that for every method that does not have a \ic{void} return type,
for every possible control-flow path through the method body,
there must be a \ic{return} statement that is eventually reached.
For example, the following method does not meet this requirement
because when \ic{arg} is \ic{false}, the \ic{return} statement is not
reached:
\begin{code}
public int twoIfTrue(bool arg) {
  if (arg) {
    return 2;
  }
}
\end{code}

\begin{example}
Consider the following method:
\begin{code}
public void mystery (bool arg0, int arg1) {
  if (arg0) {
    return -1 * arg1;
  }
  return arg0;
}
\end{code}

\noindent What do each of the following evaluate to?
\begin{itemize}
\item \ic{mystery(true, -2)}
\item \ic{mystery(true, 0)}
\item \ic{mystery(true, 4)}
\item \ic{mystery(false, -2)}
\item \ic{mystery(false, 0)}
\item \ic{mystery(false, 4)}
\end{itemize}

\noindent \emph{Answer:}
\begin{itemize}
\item \ic{mystery(true, -2)} evaluates to 2
\item \ic{mystery(true, 0)} evaluates to 0
\item \ic{mystery(true, 4)} evaluates to $-4$
\item \ic{mystery(false, -2)} evaluates to $-2$
\item \ic{mystery(false, 0)} evalutes to 0
\item \ic{mystery(false, 4)} evalutates to 4
\end{itemize}

\noindent What does the method do?

\noindent \emph{Answer:} It negates its second argument
\ic{arg1} whenever its first argument \ic{arg0} is
\ic{true} and otherwise just returns its second argument
\ic{arg1}.
\end{example}

\begin{example}
How many methods can be called by the following method?
What do you think each of their return types and parameter lists are?

\begin{code}
public void sayHello(int times, String name, String day) {
  if (shouldStop(times))
    return;
  String hello = getHelloStr(name, day);
  System.out.print.ln(hello);
  sayHello(times - 1, name, day);
}
\end{code}
\noindent \emph{Answer:}
Three methods can be called: \ic{shouldStop}, \ic{getHelloStr} and \ic{sayHello}.

The return type of \ic{shouldStop} is \ic{bool}, and it has parameter list \ic{int}.
The return type of \ic{getHelloStr} is \ic{String}, and it has parameter list \ic{String, String}.
The return type of \ic{sayHello} is \ic{void}, and it has parameter list \ic{int, String, String}.
\end{example}

\begin{example}
Write a method with signature \ic{printInOrder(int, int, int)} that takes
three \ic{int}s \ic{a}, \ic{b}, and \ic{c} as arguments and prints them out,
one per line, in increasing order. E.g. \ic{printInOrder(5, 9, 7)} should output the following:
\begin{code}
5
7
9
\end{code}

\noindent \emph{Answer:}
A possible solution is the following:
\begin{code}
public void printInOrder(int a, int b, int c) {
  int smallest = a;
  int middle = b;
  int biggest = c;
  if (smallest > middle) {
    int tmp = smallest;
    smallest = middle;
    middle = tmp;
  }
  if (smallest > biggest) {
    int tmp = smallest;
    smallest = biggest;
    biggest = tmp;
  }
  if (middle > biggest) {
    int tmp = middle;
    middle = biggest;
    biggest = tmp;
  }
  System.out.println(smallest);
  System.out.println(middle);
  System.out.println(biggest);
}
\end{code}
\end{example}

\subsection{Overloading}
As mentioned above, methods are identified by their signature, not just their names.
Because of this, we can actually have methods with the same name (in the same
Class) as long as they have different parameter lists.

For example, we can have the following two methods because
one has signature \ic{printBigger(double, double)} and the other
has signature \ic{printBigger(int, int)}:
\begin{code}
public void printBigger(double a, double b) {
  if (a > b) {
    System.out.println(a);
    return;
  }
  System.out.println(b);
}
public void printBigger(int a, int b) {
  if (a > b) {
    System.out.println(a);
    return;
  }
  System.out.println(b);
}
\end{code}
In this case, we say that the \ic{printBigger} method is \emph{overloaded}.

However, we cannot have the following two methods because they both have the
same signature \ic{twoIfTrue(bool)}:
\begin{code}
public int twoIfTrue(bool arg) {
  if (arg) {
    return 2;
  }
}
public double twoIfTrue(bool arg) {
  if (arg) {
    return 2.0;
  }
}
\end{code}

\begin{example}
Consider the following two methods that we have seen in previous
examples:
\begin{code}
private void mystery (int age, String name) {
  if (age >= 18) {
    System.out.println(name);
  }
}
public void mystery (bool arg0, int arg1) {
  if (arg0) {
    return -1 * arg1;
  }
  return arg0;
}
\end{code}
Are we allowed to have both of these methods together (in the same Class)?
Why or why not?

\noindent \emph{Answer:}
We are allowed to have both of these methods together because they have
different signatures. One has signature \ic{mystery(int, String)}
and the other has signature \ic{mystery(bool, int)}.
\end{example}

\begin{example}
Which of the following are valid overloadings?

\noindent A:
\begin{code}
public int absValue(int x) {
  if (x < 0) {
    return x * -1;
  }
  return x;
}
public double absValue(int x) {
  if (x < 0) {
    return 0.0 - x;
  }
  return x - 0.0;
}
\end{code}

\noindent B:
\begin{code}
public int absValue(int x) {
  if (x < 0) {
    return x * -1;
  }
  return x;
}
public double absValue(double x) {
  if (x < 0) {
    x = 0.0 - x;
  }
  return x;
}
\end{code}

\noindent C:
\begin{code}
public int absValue(int x) {
  if (x < 0) {
    return x * -1;
  }
  return x;
}
public int absValueX(int x) {
  if (x < 0) {
    return x * -1;
  }
  return x;
}
\end{code}

\noindent \emph{Answer:}
\begin{itemize}
\item A is not a valid overloading because both methods have the same signature.
\item B is a valid overloading because both methods have different signatures.
\item C is valid code, but this is not a valid overloading. It is not overloading at all because the methods have different names.
\end{itemize}
\end{example}

\subsection{Methods Summary}
In summary, every method definition has the following:
\begin{itemize}
\item An access modifier (if not explicitly stated, this will be the \emph{default} modifier)
\item A return type (possibly \ic{void})
\item A name
\item A (possibly empty) list of parameter types and names that can be passed to the method when it is called
\item A body
\item A return statement reached on every control path in the body, where each \ic{return} is followed by an
expression of the method's return type (unless the return type is \ic{void})
\end{itemize}
Each method is identified by its signature, so you may have methods with the same name but different parameter lists.

\section{Abstraction}\label{sec:abstraction}
Let us now return to our problem of finding the maximum scores for each
of the three groups.
The code for each of the three groups, though similar, have different variable names;
however, this is the only way in which they are different.
The process of distilling the similarity among different pieces of code is
the process of \emph{abstraction}.
In this case, we would like to \emph{abstract} away the details of
each copy to achieve a piece of code that describes all of their behavior.

After abstracting away the variable names, the code for each group is such that,
if we use the appropriate variable names for the students in the group
instead of \ic{x}, \ic{y}, and \ic{z}, that the following code snippet
would describe all of the three different operations, with
\ic{max} storing the desired result:
\begin{code}
int max = x;
if (y > max) {
  max = y; 
}
if (z > max) {
  max = z;
}
\end{code}
We can wrap this code snippet up in a method definition
that returns the value we care about:
\begin{code}
public int max3(int x, int y, int z) {
  int max = x;
  if (y > max) {
    max = y;
  }
  if (z > max) {
    max = z;
  }
  return max;
}
\end{code}

If we use this method in place of the redundant code,
our original code that compute the maximum scores for each of the three
groups then becomes as follows:
\begin{code}
int group1Max = max3(group1s1, group1s2, group1s3);
int group2Max = max3(group2s1, group2s2, group2s3);
int group3Max = max3(group3s1, group3s2, group3s3);
\end{code}
This resulting code is much more concise and is an example
of \emph{code reuse}, where the same exact code
(i.e. the code inside the body of \ic{max3}) is being reused
in several places. This code is also perhaps easier to understand
if you know that \ic{max3} simply calculates the maximum
of its three arguments. In the original code, after
going through the code for one of the groups
and realizing that it computes a maximum, you
would have to go through the calculations of the other
groups to make sure that they also compute a maximum.
Here, it is easy to see that the same computation is happening
for each group (but with different inputs).

We might further notice that both of the conditional statements
in the \ic{max3} method
compute very similar things (i.e. the maximum of two numbers),
and might perform further abstraction to achieve the following
method:
\begin{code}
int max2(int x, int y) {
  int max = x;
  if (y > max) {
    max = y; 
  }
}
\end{code}

We can then adjust our \ic{max3} method to call this one:
\begin{code}
int max3(int x, int y, int z) {
  return max2(max2(x, y), z);
}
\end{code}

Note that we do not need to change our code
for computing \ic{group1Max}, \ic{group2Max},
or \ic{group3Max} in order to use the method
\ic{max2}. The changes to \ic{max3} are sufficient.

\section{Modularity}\label{sec:modularity}
Let us again consider the case in which our calculation of the maximum
score is incorrect because we used \ic{<} instead of \ic{>} in
the last comparison. Let us assume that we are still using the
version of \ic{max3} that does not call \ic{max2}, but
we mistakenly have the comparison \ic{z < max} instead of
\ic{z > max} in our method:
\begin{code}
public int max3(int x, int y, int z) {
  int max = x;
  if (y > max) {
    max = y;
  }
  if (z < max) {
    max = z;
  }
  return max;
}
\end{code}

In order to fix, this we simply need to change \ic{z < max}
to \ic{z > max} once. The code that calculates
the values of \ic{group1Max}, \ic{group2Max},
and \ic{group3Max} by calling \ic{max3} is fixed by this
one change because for all the groups, we call the
same method. Here we have achieved a \emph{separation of concerns}
in the two parts of our code:
\begin{itemize}
\item We have one part of our code (the \ic{max3} method)
that is concerned with finding the maximum of three \emph{arbitrary}
numbers, but it is not concerned with \emph{which} numbers
specifically for which it is finding the maximum.
\item We have another part that
is concerned with computing the maximum scores for each group,
\emph{given that} we have some other code that
can calculate the maximum of three numbers, but it is
not concerned with \emph{how} this maximum is found,
as long as it is done correctly
\end{itemize}

This separation of concerns is referred to as \emph{modularity},
and we can regard the code inside of different methods as
being in different \emph{modules}.
We have already seen that modularity can allow us to do
things like fix bugs in one part of our code without having
to touch other parts of our code.

Modularity also lets us do other things, like naturally
divide up labor. If you wanted to split up the work
of writing a program that finds the maximum of each
group's three scores with a friend,
you might volunteer to write the code that deals with
finding the maximum scores for each group provided that
your friend writes code that finds the maximum of three numbers
contained in a method with signature \ic{max3(int, int int)} that returns
an \ic{int} that is the maximum of its three \ic{int} arguments.

Given that you know your friend will write such a \ic{max3}
method, you can, without seeing your friend's code,
write the following code (that we have already seen):
\begin{code}
int group1Max = max3(group1s1, group1s2, group1s3);
int group2Max = max3(group2s1, group2s2, group2s3);
int group3Max = max3(group3s1, group3s2, group3s3);
\end{code}

\noindent Once your friend finishes writing \ic{max3}, then your code
should work as expected. 

An advantage of modularity is that it is not necessary
to know the details of how the modules that you use
are implemented nor used.
In the above example, you do not need to know how exactly your
friend implements \ic{max3} (maybe it calls \ic{max2}
and maybe it does not) and your friend does not need to
know how you use \ic{max3} (maybe you also use it to
find the maximum of the three maximum scores and maybe
you do not).

\begin{example}
What would you add to the following code to calculate
the maximum of \ic{group1Max}, \ic{group2Max},
and \ic{group3Max} and store it in \ic{allGroupMax}?

\begin{code}
int group1Max = max3(group1s1, group1s2, group1s3);
int group2Max = max3(group2s1, group2s2, group2s3);
int group3Max = max3(group3s1, group3s2, group3s3);
\end{code}

\noindent \emph{Answer:}
\begin{code}
int group1Max = max3(group1s1, group1s2, group1s3);
int group2Max = max3(group2s1, group2s2, group2s3);
int group3Max = max3(group3s1, group3s2, group3s3);
// added code below:
int allGroupMax = max3(group1Max, group2Max, group3Max);
\end{code}
\end{example}

\begin{example}
Consider the case where not only do you
want to compute the maximum scores for each group
for Group 1, Group 2, and Group 3, but you
also want to do the same for Group 4 and Group 5.
Unfortunately, while Group 4 and Group 5 also have
three scores per group (\ic{group4s1}, \ic{group4s2},
\ic{group4s3}, \ic{group5s1}, \ic{group5s2},
and \ic{group5s3}), these scores are all \ic{double}s
rather than \ic{int}s.

You want write the following code:
\begin{code}
int group1Max = max3(group1s1, group1s2, group1s3);
int group2Max = max3(group2s1, group2s2, group2s3);
int group3Max = max3(group3s1, group3s2, group3s3);
double group4Max = max3(group4s1, group4s2, group4s3);
double group5Max = max3(group5s1, group5s2, group5s3);
\end{code}

What code do you ask your friend to write
so that your code works?

\noindent \emph{Answer:} You can ask your friend to write
an overloaded \ic{max3}: one overloading (the one
we have seen so far) should return an
\ic{int} and have method signature \ic{max3(int, int, int)}
and the other should return a \ic{double} and have
method signature \ic{max3(double, double, double)}.
Both versions of \ic{max3} should return the maximum value
of their three arguments.
\end{example}

\begin{example}
Give the signatures of methods that you would need to write
to make the following code work:
\begin{code}
int computeAverages(int a, int b, int c, int d, int e, int f, int g, int h) {
  int avg0 = sum(a, b) / 2;
  int avg1 = sum(c, d, e) / 3;
  int avg2 = sum(f, g, h) / 3;
  int avgAvg = sum(avg0, avg1, avg2) / 3;
  return avgAvg;
}
\end{code}

\noindent \emph{Answer:}
\ic{sum(int, int)} and \ic{sum(int, int, int)}
\end{example}

\begin{example}
This is the same code as in the previous example, but
someone made a mistake and divided by 4 instead of 3
when taking the average of three numbers:
\begin{code}
int computeAverages(int a, int b, int c, int d, int e, int f, int g, int h) {
  int avg0 = sum(a, b) / 2;
  int avg1 = sum(c, d, e) / 4;
  int avg2 = sum(f, g, h) / 4;
  int avgAvg = sum(avg0, avg1, avg2) / 4;
  return avgAvg;
}
\end{code}
Improve the original code by using abstraction.

\noindent \emph{Answer:}
\begin{code}
int avg(int a, int b) {
  return sum(a, b) / 2;
}
int avg(int a, int b, int c) {
  return sum(a, b, c) / 3;
}
int computeAverages(int a, int b, int c, int d, int e, int f, int g, int h) {
  int avg0 = avg(a, b);
  int avg1 = avg(c, d, e);
  int avg2 = avg(f, g, h);
  int avgAvg = sum(avg0, avg1, avg2) / 4;
  return avgAvg;
}
\end{code}

\end{example}
