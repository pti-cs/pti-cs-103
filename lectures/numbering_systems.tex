\chapter{Numbering Systems}
How do we represent numbers? Based on how we represent numbers, certain operations
can be easier or harder. Today, we’ll consider two representations: decimal and
binary numbers. Decimal numbers --- decimal notation or base 10 --- are what we’re
all used to seeing in our daily lives. For example we could consider the string of
numerals:

$$ 19,874 $$

And we know this represents the value nineteen thousand eight hundred seventy four.
However, we could consider representing numbers differently. Depending on how we
represent numbers specific operations become easier or harder. Let’s go back to our
example and now we’ll multiply by 10.

\begin{align*}
19,874&\\
\underline{\times\hspace{6mm}10}&\\
198,740&\\
\end{align*}

We see it is very easy to find the answer --- we only need to add a 0 at the end
of our number.

Today we’ll see how to find the value of numbers represented in decimal and
binary notation. And how to convert between the two notations. We’ll start with
the decimal notation that we see in our everyday lives.

\section{Decimal Numbers}
When we see numbers in our daily lives there’s an implicit assumption that these
numbers are decimal numbers, whether this is a price, the temperature, or time.
We can be explicit about what representation we are using by subscripting our
numbers with the base.

$$ 19,874_{10} \hspace{7mm}\text{or}\hspace{7mm}  12.38_{10}$$

In this class, unless otherwise stated we’ll consider all numbers to be
represented in base 10.

\section{Radix Decomposition}
How do we actually go from numbers in base 10 to finding their value? We can break
down the number by each position. Let’s go back to our running example:

\begin{alignat*}{12}
& 19,874_{10} &&=&& 1\times10,000 &&+&& 9\times1,000 &&+&& 8\times100  &&+&& 7\times10   &&+&& 4\times1    &\\
&             &&=&& 1\times10^4   &&+&& 9\times10^3  &&+&& 8\times10^2 &&+&& 7\times10^1 &&+&& 4\times10^0 &\\
\end{alignat*}

How do we know the value is nineteen thousand eight hundred seventy four? We can
break it down digit by digit and add together the values. We have 4 in the 1s (the
right-most position), 7 in the 10’s position, 8 in the hundreds position, 9 in the
thousands, and 1 in the ten-thousands position. Or in other words, at each
successive position we multiply the value of the digit at that position by the next
power of 10 to determine its value. This even works for non-whole numbers. Consider
the number:

\begin{alignat*}{10}
& 12.38_{10} &&=&& 1\times10   &&+&& 2\times1    &&+&& 3/10           &&+&& 8/100            &\\
&            &&=&& 1\times10^1 &&+&& 2\times10^0 &&+&& 3\times10^{-1} &&+&& 8 \times 10^{-2} &\\
\end{alignat*}

We find 1 in the tens position, 2 in the ones position, 3 in the tenths position,
and 8 in the hundredths position. For non-whole numbers we treat every digit to the
right of the decimal (or radix) point exactly the same as we do for whole numbers.
Everything to the left of the radix point we successively divide by the next power
of ten. Now that we know how to determine the value of a decimal number, how can we
do this for other number representations. We’ll now transition to considering binary
numbers.

\section{Binary Numbers}
What are binary numbers? They’re just another way to represent numbers; however,
instead of having ten digits (zero to nine) we have two bits (zero and one). Last
week, we learned about computer hardware. A computer’s primary purpose is to
compute; so we need to be able to store numbers on a computer’s hardware. Without
getting too technical, a computer represents numbers in a sequence of transistors,
each storing an electrical charge. These charges can be on (high voltage) or off
(low voltage) or somewhere inbetween. However, like with a light bulb that may burn
brighter or be dimmer based on the incoming charge, we only consider if the state
is on or off.

In effect, computers represent numbers in binary. To better understand how
computers work, we’ll practice with binary numbers. Let’s consider a few
example numbers.

$$1_2 = 1_{10} \hspace{1cm} 11_2 = 3_{10} \hspace{1cm} 101_2 = 5_{10} \hspace{1cm} 110.01_2 = 6.25_{10}$$

Here we see how we represent 1, 3, 5, and 6.25 in binary notation.

\section{Binary Arithmetic}
Just as we perform arithmetic on decimal numbers, we can perform arithmetic on
binary numbers. The process is exactly the same, except we work with bits instead of
digits. Let’s start by looking at addition and multiplication.

\noindent%
\begin{minipage}{.5\linewidth}
\begin{align*}
10110101_2&\\
\underline{+\hspace{1mm}1010110_2}&\\
100001011_2&\\
\end{align*}
\end{minipage}%
\begin{minipage}{.5\linewidth}
\begin{align*}
110101_2&\\
\underline{\times\hspace{3mm}1101_2}&\\
110101_2&\\
11010100_2&\\
\underline{+\hspace{1mm}110101000_2}&\\
1010110001_2&\\
\end{align*}
\end{minipage}

Here we see how to add the binary representations of 181 and 86 and get 267 as the
result. We also see how to do long form multiplication of 53 and 13 to get 689. You’ll
notice this is exactly the same as for decimal numbers but we force ourselves to only
work with bits, carrying the one to the next place when necessary. Both subtraction
and division work similarly.

Now we’ll turn our attention to three important arithmetic operations on binary numbers
\textbf{and}, \textbf{or}, and \textbf{exclusive or} (\textbf{xor}). These are generally
called bit-wise operations as they apply to each bit place without considering the result
of other placements. For these operations, we’ll pad the shorter of the two numbers with
zeros when needed.

\begin{center}
\begin{tabular}{|c|c|c|c|c|}
\hline
$a$ & $b$ & $a$ \textbf{and} $b$ & $a$ \textbf{or} $b$ & $a$ \textbf{xor} $b$ \\
\hline
0   &  0  &  0  &  0  &  0  \\ 
0   &  1  &  0  &  1  &  1  \\
1   &  0  &  0  &  1  &  1  \\
1   &  1  &  1  &  1  &  0  \\
\hline
\end{tabular}
\end{center}

Let’s practice a few examples!

\noindent%
\begin{minipage}{0.33333\linewidth}
\begin{align*}
101011001_2 &\\
\underline{\textbf{and}\hspace{4mm}11101_2}&\\
11001_2&\\
\end{align*}
\end{minipage}%
\begin{minipage}{0.33333\linewidth}
\begin{align*}
101010110_2 &\\
\underline{\textbf{or}\hspace{4mm}111011_2}&\\
101111111_2 &\\
\end{align*}
\end{minipage}%
\begin{minipage}{0.33333\linewidth}
\begin{align*}
1100011_2 &\\
\underline{\textbf{xor}\hspace{2mm}110110_2}&\\
1010101_2 &\\
\end{align*}
\end{minipage}

We notice that all of the operators perform the operation to each bit position
independently of the rest of the positions in the numbers.

\section{Binary to Decimal Conversion}
How do we go from binary numbers to decimal and back? We’ll now examine how to
take binary numbers and convert them to their decimal notation. This works
exactly like the decimal decomposition we learned at the beginning of class.
And in fact this works to convert a number represented in any base to decimal.

\begin{alignat*}{12}
& 1010110001_2 &&=&& 2^9      &&+&& 2^7 &&+&& 2^5 &&+&& 2^4 &&+&& 2^0 & \\
&              &&=&& 512      &&+&& 128 &&+&& 32  &&+&& 16  &&+&& 1   & \\
&              &&=&& 689_{10} && &&     && &&     && &&     && &&     & \\
\end{alignat*}

\section{Decimal to Binary Conversion}
Now, let’s look at the opposite conversion, converting decimal numbers to
binary. This will use successive division and subtraction as opposed to
addition in multiplication. We will successively divide by two, if the
remainder is non-zero we’ll keep the bit at the current position. We’ll
continue until we can no longer divide by two. Again, this works for any
base, not just base two.

\begin{center}
\begin{tabular}{|c|c|c|}
\hline
\textbf{Initial Value} & \textbf{Result} & \textbf{Remainder} \\
\hline
267 & 133 & 1 \\
133 &  66 & 1 \\
 66 &  33 & 0 \\
 33 &  16 & 1 \\
 16 &   8 & 0 \\
  8 &   4 & 0 \\
  4 &   2 & 0 \\
  2 &   1 & 0 \\
  1 &   0 & 1 \\
\hline
\end{tabular}
\end{center}

$$267_{10} = 100001011_2$$

\section{Conclusion}
Today, we covered two very important number representations. In particular,
decimal notation that we’re used to seeing in everyday life and binary notation
that is useful for understanding how computers manipulate numbers. We covered
binary addition, multiplication, and three bitwise arithmetic operations
(\textbf{and}, \textbf{or}, and \textbf{xor}). Then we showed how we can convert
between binary and decimal notation for numbers. Next lecture, we’ll review what
we’ve learned about binary numbers and introduce Hexadecimal numbers, another
important number representation in computer science.

\section{Learning Objectives}
After covering this chapter, you should be able to answer the following questions:
\begin{enumerate}
\item How many digits are in $142,123,098_{10}$?
\item How many bits are in $100101110010101_2$?
\item Convert $421_{10}$ to binary.
\item How many bits do you need to represent $123,042,982_{10}$?
\item How many digits are required to represent $10110110_2$?
\end{enumerate}
